\section{Conclusion}

We have presented HotSpotter, an algorithm for fast,
reliable, multi-species animal identification based on
extracting and matching keypoints and descriptors.  In addition to the
construction and testing of the overall system, the primary technical
contribution of our work is the development of a fast and scalable
one-vs-many scoring mechanism which outperforms current brute force
one-vs-one image comparison methods in both speed and accuracy.  The
key to the success of HotSpotter is the use of viewpoint invariant
descriptors and a scoring mechanism that emphasizes the most
distinctiveness keypoints and descriptors by allowing only the $k$
nearest neighbors of any descriptor to participate in the scoring.
This has been borne out on experiments with Grevy's zebras, plains
zebras, giraffes, leopards and lionfish.

From the perspective of the application, the failures are generally
manageable through human interaction to eliminate problems due to (a)
ROIs that overlap multiple animals, (b) matching
against background, and (c) poor quality images. Additional algorithms
could help identify and eliminate these problems as well.  Other
failures --- those due to viewpoint variations --- are the most difficult to handle. The
best way to address this is to incorporate multiple images and
viewpoints into the database for each animal.

Looking toward future experimental work, we will  
  apply the software to a wider variety of species, which will test the limit 
  recognition based on distinctive feature matching. From the perspective 
of both the algorithm and the practical application,
 the next major step for HotSpotter is to build and manage a dynamically
constructed database of animal images and labels.  This is the focus of
our ongoing effort.


%  We have evaluated several flavors of scoring using techniques based on the local naive Bayes nearest neighbor framework and found comparable performance across the board.

%  We briefly discussed and evaluated a method for extreme memory scalability using product quantization
%  and show that excellent results can still be achieved by matching to multiple nearest neighbors.

%  During our research we encountered the question: "How do we build a dynamic image database which can be searched quickly?" We leave this issue open to be addressed in future work.
